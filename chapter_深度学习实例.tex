\chapter{深度学习\cite{Lecun2015Deeplearning}}
\section{CNN实例}
CNN(Convolutional Neural Network)是深度学习算法在图像处理领域的一个应用。

利用卷积神经网络将一幅图像的内容与另一幅图像的风格相结合。~\cite{Johnson2015}。我在一个博客中看到了一篇原理分析的文章。
http://blog.csdn.net/Gavin__Zhou/article/details/53144148

深度学习是一种方法,神经网络是个模型。用了深度学习可以有效解决层数多的网络不好学习的问题。

神经网络算法是实现深度学习的一种算法,深度学习是实现人工智能的一种方式。

深度学习可以理解成用深度神经网络(DNN,Deep Neural Network)来进行机器学习,他俩的关系可以从这个定义中一目了然地看出来。深度神经网络(DNN)一般分为三种架构:
    
朴素的DNN:就是一般性的神经网络往多层扩展,缺点很多包括训练缓慢,用Backprop进行训练梯度衰减得厉害;
深度置信(信念)网络(DBN,Deep Belief Network):基于RBN(Restricted Boltzmann Machine)的性质而建立起来的深度神经网络,优点是比朴素的DNN训练快些,适用于最大似然概率的估计;
卷积深度置信网络(CDBN,Convolutional Deep Belief Networks):比DBN训练更快些,适用于非常大型的图像或者语音识别。
    
困难一个是训练速度,另一个就是需要大量的计算力啦,这个成本挺高的,像Google的AlphaGo用了1202个CPU+176个GPU,普通的人和公司也拿不出这么强的计算力。