\documentclass[a4paper,12pt]{ctexbook}
\usepackage[margin=1cm]{geometry}
\usepackage{graphicx}
\usepackage{subfigure}
\usepackage{float}
\usepackage[colorlinks,linkcolor=black]{hyperref}%colorlinks启用链接颜色,linkcolor指定对应的颜色
\usepackage{listings}
%\usepackage[cache=false]{minted}  % 代码高亮X
\usepackage{fontspec}
\usepackage{tikz,xcolor,mwe}

\definecolor{cvgreen}{HTML}{92D14F}
\definecolor{cvgray}{HTML}{D8E4BE}
\definecolor{cvtext}{HTML}{92909B}
\usetikzlibrary{shadows}

\pagestyle{empty}
\CTEXsetup[format={\large\bfseries}]{section}%可以让section的标题左对齐。
\CTEXsetup[format={\Large\bfseries}]{chapter}%可以让chapter的标题左对齐。
%\CTEXsetup[format+={\flushleft}]{section}%让section的标题居左
%\renewcommand{\thesection}{\chinese{section}}%将“1.1”改为汉字“一”,但是subsection就会变成  六.1 ,比较难看,还是不用比较好。

\begin{document}


\begin{tikzpicture}[remember picture,overlay]
\fill[cvgreen] (current page.north west) rectangle ([xshift=6cm]current page.south west);% green bar
\fill[cvgray] ([yshift=-11cm]current page.north west) rectangle ([yshift=-17cm]current page.north east); % gray bar

\node[cvtext,right] at ([xshift=3.5cm,yshift=-13cm]current page.north west) {\Huge Machine Learning way};
\node[cvtext,above left] at ([xshift=-1cm,yshift=-16.5cm]current page.north east) {\Large\bfseries \today};
% cover photo
\node[inner sep=0pt,below right] (image) at ([xshift=17cm,yshift=-1cm]current page.north west) {\includegraphics[width=3cm]{hainulogo.png}};
% name and address
\node[fill=white,align=center,text width=6.4cm,inner sep=0.8cm,below] (name) at (image.south) {};
\node[text width=15cm,inner sep=0.3cm,below right] at (name.south west){\Large Author:Flynn\\E-Mail:zofon@qq.com};
% attachments
\node[white,text width=5cm,inner sep=0.6cm,above right] at ([yshift=1cm]current page.south west)
{\large\obeylines\textbf{This is\\Open Source}};
\end{tikzpicture}

\tableofcontents

\chapter{相关的概念}
在学习机器学习之前首先要了解一些相关的概念。\cite{Johnson2015}
\begin{description}
  \item[AI] 人工智能是最早出现的,也是最大、最外侧的同心圆;\cite{深度学习框架的评估与比较_孙镜涛}
  \item[ML] 机器学习
  \item[DL] 深度学习,深度学习造成了前所未有的巨大的影响, 是当今人工智能大爆炸的核心驱动。
\end{description}
五十年代,人工智能曾一度被极为看好。之后,人工智能的一些较小的子集发展了起来。先是机器学习,然后是深度学习。深度学习又是机器学习的子集。。

早在1956年夏天那次会议,人工智能的先驱们就梦想着用当时刚刚出现的计算机来构造复杂的、拥有与人类智慧同样本质特性的机器。这就是我们现在所说的“强人工智能”(General AI)。这个无所不能的机器,它有着我们所有的感知(甚至比人更多),我们所有的理性,可以像我们一样思考。

人们在电影里也总是看到这样的机器:友好的,像星球大战中的C-3PO;邪恶的,如终结者。强人工智能现在还只存在于电影和科幻小说中,原因不难理解,我们还没法实现它们,至少目前还不行。

我们目前能实现的,一般被称为“弱人工智能”(Narrow AI)。弱人工智能是能够与人一样,甚至比人更好地执行特定任务的技术。例如,Pinterest上的图像分类;或者Facebook的人脸识别。

这些是弱人工智能在实践中的例子。这些技术实现的是人类智能的一些具体的局部。但它们是如何实现的?这种智能是从何而来?这就带我们来到同心圆的里面一层,机器学习。

\chapter{如何研究学习一个机器学习算法}
中文:\url{http://blog.jobbole.com/80658/}
英文:\url{http://machinelearningmastery.com/how-to-investigate-machine-learning-algorithm-behavior/}
\section{引言}
机器学习算法都是一个个复杂的体系,需要通过研究来理解。学习算法的静态描述是一个好的开始,但是这并不足以使我们理解算法的行为,我们需要在动态中来理解算法。

机器学习算法的运行实验,会使你对于不同类型问题得出的实验结论,并对实验结论与算法参数两者的因果关系有一个直观认识。

在这篇文章中,你将会知道怎么研究学习一个机器学习算法。你将会学到5个简单步骤,你可以用来设计和完成你的第一个机器学习算法实验

你会发现机器学习实验不光是学者们的专利,你也可以;你也会知道实验是通往精通的必经之路,因为你可以从经验中学到因果关系的知识, 这是其它地方学不到的。

\section{什么是研究机器学习算法}
当研究一个机器学习算法的时候,你的目标是找到可得到好结果的机器算法行为,这些结果是可以推广到多个问题或者多个类型的问题上。

你通过对算法状态做系统研究来研究学习机器学习算法。这项工作通过设计和运行可控实验来完成

一旦你完成了一项实验,你可以对结论作出解释和提交。这些结论会让你得以管窥在算法变化中因果关系。这就是算法行为和你获得的结论间的关系。

\section{怎样研究学习机器学习算法}
在这一部分,我们将学到5个简单的步骤,你可以通过它来研究学习一个机器算法
\subsection{选择一个算法}
选择一个你有疑问的算法

这个算法可能是你正在某个问题上应用的,或者你发现在其他环境中表现很好,将来你想使用

就实验的意图来说,使用现成的算法是有帮助的。这会给你一个底线:存在bug几率最低

自己实现一个算法可能是了解算法过程的一个好的方式,但是,实验期间,会引入额外的变量,比如bug,和大量必须为算法所做的微观决策
\subsection{确定一个问题}
你必须有一个你试图寻找答案的研究问题。问题越明确,问题越有用

给出的示例问题包括以下几个方面:

KNN算法中,作为样本空间中的一部分的K值在增大时有什么影响?

在SVM算法中,选择不同的核函数在二分类问题上有什么影响 ?

在二分类问题中,逻辑回归上的不同参数的缩放有什么影响 ?

在随机森林模型中,在训练集上增加任意属性对在分类准确性上有什么影响?

针对算法,设计你想回答的问题。仔细考虑,然后列出5个逐渐演变的问题,并且深入推敲那个最精确的
\subsection{设计实验}
从你的问题中挑选出关键元素然后组成你的实验内容。 例如,拿上面的示例问题为例:“二元分类问题中逻辑回归上的不同的参数缩放有什么影响?”

你从这个问题中挑出来用来设计实验的元素是:

属性缩放法:你可以采用像正态化、标准化,将某一属性提升至乘方、取对数等方法

逻辑回归:你想使用哪种已经实现的逻辑回归。

二元分类问题:存在数值属性不同的二分类问题标准。需要准备多种问题,其中一些问题的规模是相同的(像电离层),然而其他一些问题的属性有不同的缩放值(像糖尿病问题)。

性能: 类似分类准确性的模型性能分数是需要的

花时间仔细挑选你问题中的组成元素以便为你的问题给出最佳解答。
\subsection{进行试验并且报告你的结论}
完成你的实验

如果算法是随机的,你需要多次重复实验操作并且记录一个平均数和标准偏差

如果你试图寻找在不同实验(比如带有不同的参数)之间结果的差异,你可能想要使用一种统计工具来标明差异是否统计上显著的(就像学生的t检验)

一些工具像R和scikit-learn/SciPy完成这些类型的实验,但是你需要把它们组合在一起,并且为实验写脚本。其他工具像Weka带有图形用户界面,你所使用的工具不要影响问题和你实验设计的严密

总结你的实验结论。你可能想使用图表。单独呈现结果是不够的,他们只是数字。你必须将数字和问题联系起来,并且通过你的实验设计提取出它们的意义

对实验问题来说,实验结果又暗示着什么呢?

保持怀疑的态度。你的结论上有留什么样的漏洞和局限呢。不要逃避这一部分。知道局限性和知道实验结果一样重要
\subsection{重复}
重复操作

继续研究你选择的算法。你甚至想要重复带有不同参数或者不同的测试数据集的同一个实验。你可能想要处理你试验中的局限性

不要只停留在一个算法上,开始建立知识体系和对算法的直觉

通过使用一些简单工具,提出好的问题,保持严谨和怀疑的态度,你对机器算法行为的理解很快就会到达世界级的水平
\section{研究学习算法不仅仅是学者才能做的}
你也可以学习研究机器学习算法。

你不需要一个很高的学位,你不需要用研究的方式训练,你也不需要成为一名学者

对每个拥有计算机和浓厚兴趣的人来说,机器学习算法的系统研究学习是开放的。事实上,如果你主修机器学习,你一定会适应机器学习算法的系统研究。知识根本不会自己出来,你需要靠自己的经验去得到

当谈论你的发现的适用性时,你需要保持怀疑和谨慎

你不一定提出独一无二的问题。通过研究一般的问题,你也将会收获很多,例如根据一些一般的标准数据集总结出一个参数的普遍影响。你保不住会发现某些具有最优方法的常例的局限性甚至反例。

\include{chapter_框架比较}

\chapter{神经网络入门}

文章参考:\url{http://blog.csdn.net/zzwu/article/details/574931}

\section{生物学的大脑}
大脑是一块灰色的、像奶冻一样的东西。它并不像脑中的CPU那样,利用单个或少数几个处理单元来进行工作。如果你有一具新鲜地保存到福尔马林中的尸体,用一把锯子小心地将它的头骨锯开,搬掉头盖骨后,你就能看到熟悉的脑组织皱纹。大脑的外层象一个大核桃那样,全部都是起皱的[图0左],这一层组织就称皮层(Cortex)。如果你再小心地用手指把整个大脑从头颅中端出来,再去拿一把外科医生用的手术刀,将大脑切成片,那么你将看到大脑有两层: 灰色的外层(这就是“灰质”一词的来源,但没有经过福尔马林固定的新鲜大脑实际是粉红色的。如图~\ref{大脑皮层解剖图}) 和白色的内层。灰色层只有几毫米厚,其中紧密地压缩着几十亿个被称作neuron(神经细胞、神经元)的微小细胞。白色层在皮层灰质的下面,占据了皮层的大部分空间,是由神经细胞相互之间的无数连接线组成(但没有神经细胞本身,正如印刷电路板的背面,只有元件的连线,而没有元件本身那样,译注)。皮层象核桃一样起皱,这可以把一个很大的表面区域塞进到一个较小的空间里。这与光滑的皮层相比能容纳更多的神经细胞。人的大脑大约含有10G(即100亿)个这样的微小处理单元;一只蚂蚁的大脑大约也有250,000个。
\begin{figure}[H]
  \centering
  \includegraphics[width=10cm]{Figures/大脑皮层解剖图.jpg}
  \caption{大脑皮层解剖图}\label{大脑皮层解剖图}
\end{figure}



\chapter{深度学习\cite{Lecun2015Deeplearning}}
\section{CNN实例}
CNN(Convolutional Neural Network)是深度学习算法在图像处理领域的一个应用。

利用卷积神经网络将一幅图像的内容与另一幅图像的风格相结合。~\cite{Johnson2015}。我在一个博客中看到了一篇原理分析的文章。
http://blog.csdn.net/Gavin__Zhou/article/details/53144148

深度学习是一种方法,神经网络是个模型。用了深度学习可以有效解决层数多的网络不好学习的问题。

神经网络算法是实现深度学习的一种算法,深度学习是实现人工智能的一种方式。

深度学习可以理解成用深度神经网络(DNN,Deep Neural Network)来进行机器学习,他俩的关系可以从这个定义中一目了然地看出来。深度神经网络(DNN)一般分为三种架构:
    
朴素的DNN:就是一般性的神经网络往多层扩展,缺点很多包括训练缓慢,用Backprop进行训练梯度衰减得厉害;
深度置信(信念)网络(DBN,Deep Belief Network):基于RBN(Restricted Boltzmann Machine)的性质而建立起来的深度神经网络,优点是比朴素的DNN训练快些,适用于最大似然概率的估计;
卷积深度置信网络(CDBN,Convolutional Deep Belief Networks):比DBN训练更快些,适用于非常大型的图像或者语音识别。
    
困难一个是训练速度,另一个就是需要大量的计算力啦,这个成本挺高的,像Google的AlphaGo用了1202个CPU+176个GPU,普通的人和公司也拿不出这么强的计算力。

\bibliographystyle{plain}
\bibliography{Citations}
\end{document}
