\chapter{如何研究学习一个机器学习算法}
中文:\url{http://blog.jobbole.com/80658/}
英文:\url{http://machinelearningmastery.com/how-to-investigate-machine-learning-algorithm-behavior/}
\section{引言}
机器学习算法都是一个个复杂的体系,需要通过研究来理解。学习算法的静态描述是一个好的开始,但是这并不足以使我们理解算法的行为,我们需要在动态中来理解算法。

机器学习算法的运行实验,会使你对于不同类型问题得出的实验结论,并对实验结论与算法参数两者的因果关系有一个直观认识。

在这篇文章中,你将会知道怎么研究学习一个机器学习算法。你将会学到5个简单步骤,你可以用来设计和完成你的第一个机器学习算法实验

你会发现机器学习实验不光是学者们的专利,你也可以;你也会知道实验是通往精通的必经之路,因为你可以从经验中学到因果关系的知识, 这是其它地方学不到的。

\section{什么是研究机器学习算法}
当研究一个机器学习算法的时候,你的目标是找到可得到好结果的机器算法行为,这些结果是可以推广到多个问题或者多个类型的问题上。

你通过对算法状态做系统研究来研究学习机器学习算法。这项工作通过设计和运行可控实验来完成

一旦你完成了一项实验,你可以对结论作出解释和提交。这些结论会让你得以管窥在算法变化中因果关系。这就是算法行为和你获得的结论间的关系。

\section{怎样研究学习机器学习算法}
在这一部分,我们将学到5个简单的步骤,你可以通过它来研究学习一个机器算法
\subsection{选择一个算法}
选择一个你有疑问的算法

这个算法可能是你正在某个问题上应用的,或者你发现在其他环境中表现很好,将来你想使用

就实验的意图来说,使用现成的算法是有帮助的。这会给你一个底线:存在bug几率最低

自己实现一个算法可能是了解算法过程的一个好的方式,但是,实验期间,会引入额外的变量,比如bug,和大量必须为算法所做的微观决策
\subsection{确定一个问题}
你必须有一个你试图寻找答案的研究问题。问题越明确,问题越有用

给出的示例问题包括以下几个方面:

KNN算法中,作为样本空间中的一部分的K值在增大时有什么影响?

在SVM算法中,选择不同的核函数在二分类问题上有什么影响 ?

在二分类问题中,逻辑回归上的不同参数的缩放有什么影响 ?

在随机森林模型中,在训练集上增加任意属性对在分类准确性上有什么影响?

针对算法,设计你想回答的问题。仔细考虑,然后列出5个逐渐演变的问题,并且深入推敲那个最精确的
\subsection{设计实验}
从你的问题中挑选出关键元素然后组成你的实验内容。 例如,拿上面的示例问题为例:“二元分类问题中逻辑回归上的不同的参数缩放有什么影响?”

你从这个问题中挑出来用来设计实验的元素是:

属性缩放法:你可以采用像正态化、标准化,将某一属性提升至乘方、取对数等方法

逻辑回归:你想使用哪种已经实现的逻辑回归。

二元分类问题:存在数值属性不同的二分类问题标准。需要准备多种问题,其中一些问题的规模是相同的(像电离层),然而其他一些问题的属性有不同的缩放值(像糖尿病问题)。

性能: 类似分类准确性的模型性能分数是需要的

花时间仔细挑选你问题中的组成元素以便为你的问题给出最佳解答。
\subsection{进行试验并且报告你的结论}
完成你的实验

如果算法是随机的,你需要多次重复实验操作并且记录一个平均数和标准偏差

如果你试图寻找在不同实验(比如带有不同的参数)之间结果的差异,你可能想要使用一种统计工具来标明差异是否统计上显著的(就像学生的t检验)

一些工具像R和scikit-learn/SciPy完成这些类型的实验,但是你需要把它们组合在一起,并且为实验写脚本。其他工具像Weka带有图形用户界面,你所使用的工具不要影响问题和你实验设计的严密

总结你的实验结论。你可能想使用图表。单独呈现结果是不够的,他们只是数字。你必须将数字和问题联系起来,并且通过你的实验设计提取出它们的意义

对实验问题来说,实验结果又暗示着什么呢?

保持怀疑的态度。你的结论上有留什么样的漏洞和局限呢。不要逃避这一部分。知道局限性和知道实验结果一样重要
\subsection{重复}
重复操作

继续研究你选择的算法。你甚至想要重复带有不同参数或者不同的测试数据集的同一个实验。你可能想要处理你试验中的局限性

不要只停留在一个算法上,开始建立知识体系和对算法的直觉

通过使用一些简单工具,提出好的问题,保持严谨和怀疑的态度,你对机器算法行为的理解很快就会到达世界级的水平
\section{研究学习算法不仅仅是学者才能做的}
你也可以学习研究机器学习算法。

你不需要一个很高的学位,你不需要用研究的方式训练,你也不需要成为一名学者

对每个拥有计算机和浓厚兴趣的人来说,机器学习算法的系统研究学习是开放的。事实上,如果你主修机器学习,你一定会适应机器学习算法的系统研究。知识根本不会自己出来,你需要靠自己的经验去得到

当谈论你的发现的适用性时,你需要保持怀疑和谨慎

你不一定提出独一无二的问题。通过研究一般的问题,你也将会收获很多,例如根据一些一般的标准数据集总结出一个参数的普遍影响。你保不住会发现某些具有最优方法的常例的局限性甚至反例。