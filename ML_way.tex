

在学习机器学习之前首先要了解一些相关的概念。

AI,机器学习,深度学习三者的关系。

如上图,人工智能是最早出现的,也是最大、最外侧的同心圆;其次是机器学习,稍晚一点;最内侧,是深度学习,当今人工智能大爆炸的核心驱动。

五十年代,人工智能曾一度被极为看好。之后,人工智能的一些较小的子集发展了起来。先是机器学习,然后是深度学习。深度学习又是机器学习的子集。深度学习造成了前所未有的巨大的影响。

早在1956年夏天那次会议,人工智能的先驱们就梦想着用当时刚刚出现的计算机来构造复杂的、拥有与人类智慧同样本质特性的机器。这就是我们现在所说的“强人工智能”(General AI)。这个无所不能的机器,它有着我们所有的感知(甚至比人更多),我们所有的理性,可以像我们一样思考。

人们在电影里也总是看到这样的机器:友好的,像星球大战中的C-3PO;邪恶的,如终结者。强人工智能现在还只存在于电影和科幻小说中,原因不难理解,我们还没法实现它们,至少目前还不行。

我们目前能实现的,一般被称为“弱人工智能”(Narrow AI)。弱人工智能是能够与人一样,甚至比人更好地执行特定任务的技术。例如,Pinterest上的图像分类;或者Facebook的人脸识别。

这些是弱人工智能在实践中的例子。这些技术实现的是人类智能的一些具体的局部。但它们是如何实现的?这种智能是从何而来?这就带我们来到同心圆的里面一层,机器学习。
